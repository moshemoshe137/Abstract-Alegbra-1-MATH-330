
\documentclass{article}
\usepackage[utf8]{inputenc}
\usepackage{amsmath}
\usepackage[twoside, margin=1cm]{geometry}
\usepackage[english]{babel} % English language/hyphenation
\usepackage{amsmath,amsthm,amssymb}
\usepackage{setspace} %for Double Spacing
\usepackage{breqn}
\usepackage{enumerate}
\usepackage{multicol}
%\usepackage{pgfplots}


\theoremstyle{definition}
\newtheorem{theorem}{Exercise}[section]

\newcommand{\R}{\mathbb{R}}
\newcommand{\Z}{\mathbb{Z}}
\newcommand{\N}{\mathbb{N}}
\newcommand{\C}{\mathbb{C}}
\newcommand{\inv}[1]{#1^{-1}}
\DeclareMathOperator{\ord}{ord}
%\DeclareMathOperator{\ker}{ker}
\setcounter{section}{9}

%\doublespacing
\onehalfspacing
\begin{document}
	\begin{flushright}
		Moshe Mason Rubin\\MATH 330 Homework \#8\\7 November 2016
	\end{flushright}




	\setcounter{theorem}{8}
	\begin{theorem}
		Let $G = \R\setminus\left\{-1\right\}$ and define a binary operation on $G$ by \[a\ast b = a+b+ab\text{.}\] Prove that $G$ is a group under this operation. Show that $\left(G,\ast\right)$ is isomorphic to the multiplicative group of nonzero real numbers.
	\end{theorem}
	\begin{proof}
		By \textbf{Exercise 3.7} from homework \#2, $\left(G,\ast\right)$ is an abelian group.\\
		The map $\phi:G\to \R^*$ given by $\phi(x)=1+x$ is clearly a bijection and well defined on each set. $\phi$ preserves group operations as for any $a,b\in G$, 
		\begin{dmath*}
			\phi(a) \cdot \phi(b) = \left(1+a\right) \cdot \left(1+b\right) \condition[]{by definition of $\phi$} = 1 + b + a + ab \hiderel{=} a + b + ab + 1 = \phi(a+b+ab) \condition[]{by definition of $\phi$} = \phi(a\ast b) \condition[]{by definition of $\ast$}
		\end{dmath*}
		So $\left(G, \ast\right) \simeq \left(\R^*,\cdot\right)$.
	\end{proof}

	
	\setcounter{theorem}{11}
	\begin{theorem}
		Prove that $S_4$ is not isomorphic to $D_{12}$.
	\end{theorem}
	\begin{proof}
		Consider $(1, 12, 11, 10, 9, 8, 7, 6, 5, 4, 3, 2) \in D_{12}$ which has order $12$. Because every element of $S_4$ has order less than or equal to $4$, the two groups cannot be isomorphic by Theorem from class that $\ord \left[\phi\left(g_1\right)\right] = \ord \left(g_1\right)$. 
	\end{proof}

	\setcounter{theorem}{13}
	\begin{theorem}
		Show that the set of all matrices of the form $\left[\begin{smallmatrix} \pm1&k\\0&1\end{smallmatrix}\right]$ is isomorphic to $D_n$ where all entries in the matrix are in $\Z_n$. 
	\end{theorem}
	\begin{proof}
		Let $S = \left\{\left[\begin{smallmatrix} \pm1&k\\0&1\end{smallmatrix}\right] : k\in\Z_n\right\}$. Notice for any $\left[\begin{smallmatrix} \pm1&k\\0&1\end{smallmatrix}\right]\in S$, we have \[\begin{bmatrix}
		\pm1 & k \\ 
		0 & 1
		\end{bmatrix} = \begin{bmatrix}
		1 & k \\ 
		0 & 1
		\end{bmatrix} \begin{bmatrix}
		\pm1 & 0 \\ 
		0 & 1
		\end{bmatrix} = \begin{bmatrix}
		1 & 1 \\ 
		0 & 1
		\end{bmatrix}^k \begin{bmatrix}
		\pm1 & 0 \\ 
		0 & 1
		\end{bmatrix} \] and $\left[\begin{smallmatrix} -1 & 0 \\ 0 & 1 \end{smallmatrix}\right]^2 = \left[\begin{smallmatrix} 1 & 0 \\ 0 & 1 \end{smallmatrix}\right]$. So $\left[\begin{smallmatrix} 1 & 1 \\ 0 & 1 \end{smallmatrix}\right]$ and $\left[\begin{smallmatrix} -1 & 0 \\ 0 & 1 \end{smallmatrix}\right]$ generate $S$. Furthermore, $\left[\begin{smallmatrix} 1 & 1 \\ 0 & 1 \end{smallmatrix}\right]$ has order $n$, and $\left[\begin{smallmatrix} -1 & 0 \\ 0 & 1 \end{smallmatrix}\right]$ has order $2$, and \[ \left[\begin{matrix} -1 & 0 \\ 0 & 1 \end{matrix}\right] \left[\begin{matrix} 1 & 1 \\ 0 & 1 \end{matrix}\right] \inv{\left[\begin{matrix} -1 & 0 \\ 0 & 1 \end{matrix}\right]} = \left[\begin{matrix} -1 & 0 \\ 0 & 1 \end{matrix}\right] \left[\begin{matrix} 1 & 1 \\ 0 & 1 \end{matrix}\right] \left[\begin{matrix} -1 & 0 \\ 0 & 1 \end{matrix}\right] = \left[\begin{matrix} 1 & -1 \\ 0 & 1 \end{matrix} \right] = \inv{\begin{bmatrix} 1 & 1 \\ 0 & 1 \end{bmatrix}}\]
		By Theorem 5.10, $D_n$ is generated by all the products of $r,s\in D_n$ such that $r^n = s^2 = id$ and $srs = \inv{r}$. Define $f: S\to D_n$ by $f\left( \left[ \begin{smallmatrix} 1&k\\0&1\end{smallmatrix} \right] \right) = r^k$ and $f\left( \left[ \begin{smallmatrix} -1&k\\0&1\end{smallmatrix} \right] \right) = r^{k}s$. To check that $f$ preserves group operations, there are four cases to check:
		\begin{enumerate}
			\item $f\left( \left[ \begin{smallmatrix} 1 & k \\ 0 & 1 \end{smallmatrix}  \right] \left[ \begin{smallmatrix} 1 & \ell \\ 0 & 1 \end{smallmatrix}  \right] \right) = f\left(\left[ \begin{smallmatrix} 1 & k+\ell \\ 0 & 1 \end{smallmatrix}  \right]\right) = r^{k+\ell} = r^k r^{\ell} = f\left( \left[ \begin{smallmatrix} 1 & k \\ 0 & 1 \end{smallmatrix}  \right]\right) \cdot f\left( \left[ \begin{smallmatrix} 1 & \ell \\ 0 & 1 \end{smallmatrix}  \right]\right)$. \checkmark
			
			\item $f\left( \left[ \begin{smallmatrix} -1 & k \\ 0 & 1 \end{smallmatrix}  \right] \left[ \begin{smallmatrix} 1 & \ell \\ 0 & 1 \end{smallmatrix}  \right] \right) = f\left(\left[ \begin{smallmatrix} -1 & k-\ell \\ 0 & 1 \end{smallmatrix}  \right]\right) = r^{k-\ell}s = r^k r^{-\ell}s = \cdots = \cdots = f\left( \left[ \begin{smallmatrix} -1 & k \\ 0 & 1 \end{smallmatrix}  \right]\right) \cdot f\left( \left[ \begin{smallmatrix} 1 & \ell \\ 0 & 1 \end{smallmatrix}  \right]\right)$. \checkmark
			
			\item $f\left( \left[ \begin{smallmatrix} 1 & k \\ 0 & 1 \end{smallmatrix}  \right] \left[ \begin{smallmatrix} -1 & \ell \\ 0 & 1 \end{smallmatrix}  \right] \right) = f\left(\left[ \begin{smallmatrix} -1 & k+\ell \\ 0 & 1 \end{smallmatrix}  \right]\right) = r^{k+\ell}s = r^k r^{\ell}s = r^k\left(r^{\ell} s \right) = f\left( \left[ \begin{smallmatrix} 1 & k \\ 0 & 1 \end{smallmatrix}  \right]\right) \cdot f\left( \left[ \begin{smallmatrix} -1 & \ell \\ 0 & 1 \end{smallmatrix}  \right]\right)$. \checkmark
			
			\item $f\left( \left[ \begin{smallmatrix} -1 & k \\ 0 & 1 \end{smallmatrix}  \right] \left[ \begin{smallmatrix} -1 & \ell \\ 0 & 1 \end{smallmatrix}  \right] \right) = f\left(\left[ \begin{smallmatrix} 1 & k-\ell \\ 0 & 1 \end{smallmatrix}  \right]\right) = r^{k-\ell} = r^k r^{-\ell} = r \left(srs\right)^{\ell} \cdots = \cdots = f\left( \left[ \begin{smallmatrix} -1 & k \\ 0 & 1 \end{smallmatrix}  \right]\right) \cdot f\left( \left[ \begin{smallmatrix} 1 & \ell \\ 0 & 1 \end{smallmatrix}  \right]\right)$. \checkmark
		\end{enumerate}
	\end{proof}

	
	\setcounter{section}{10}
	\setcounter{theorem}{1}
	\begin{theorem}
		Find all the sub-groups of $D_4$. Which sub-groups are normal? What are all the factors groups of $D_4$ up to isomorphisms?
	\end{theorem}
	\begin{proof}
		The sub-groups are $D_4 = \left\{ id, \rho, \rho^2, \rho^3, s, \rho s, \rho^2 s, \rho^3 s \right\} = \left\{ id, (1234), (13)(24), (1432), (24), (12)(34), (13), (14)(23) \right\}$ are \nolinebreak[4]
		\begin{enumerate}\begin{multicols}{5}
			\item $\left\{id\right\}$
			\item $\left\{id, \rho^2\right\}$
			\item $\left\{id, s\right\}$
			\item $\left\{id, \rho s\right\}$
			\item $\left\{id, \rho^2 s\right\}$
			\item $\left\{id, \rho^3 s\right\}$
			\item $\left\{id, \rho, \rho^2, \rho^3 \right\}$
			\item $\left\{id, \rho^2, s, \rho^2 s\right\}$
			\item $\left\{id, \rho^2, \rho s, \rho^3 s \right\}$
			\item $D_4$
			\end{multicols}
			\setcounter{enumi}{0}
			%\begin{multicols}{2}
			\item $\left\{ id \right\}$ is normal.
				
			\item $\left\{id, \rho^2\right\}$ is normal as $(1234)\left\{id, \rho^2\right\} = \left\{(1234),(1432)\right\} = \left\{id, \rho^2\right\}(1234)$ and $(24)\left\{id, \rho^2\right\} = \left\{(24), (13)\right\} = \left\{id, \rho^2\right\}(24)$ and $(12)(34)\left\{id, \rho^2\right\} = \left\{ (12)(34), (14)(23)\right\} = \left\{id, \rho^2\right\}(12)(34)$. \checkmark
				
			\item $\left\{id, s\right\}$ is not normal as $ (1234)\left\{id, s \right\} = \left\{(1234),(12)(34)\right\} \not= \left\{(1234),(14)(23)\right\} = \left\{id, s\right\}(1234)$.
				
			\item $\left\{id, \rho s\right\}$ is not normal as $(1234)\left\{id, \rho s\right\} = \left\{ (1234), (13) \right\} \not= \left\{ (1234), (24) \right\} = \left\{id, \rho s\right\}(1234)$
				
			\item $\left\{id, \rho^2 s\right\}$ is not normal as $\rho \circ \left(\rho^2 s\right) = \rho^3 s \not= \rho s = \left(\rho^2 s\right)\circ \rho$.
				
			\item $\left\{id, \rho^3 s\right\}$ is not normal as $\rho \circ \left(\rho^3 s\right) = s \not= \rho^2 s = \left(\rho^3 s\right) \circ \rho$.
				
			\item $\left\{id, \rho, \rho^2, \rho^3 \right\}$ is normal as $s \left\{id, \rho, \rho^2, \rho^3 \right\} = \left\{s, \rho^3 s, \rho^2 s, \rho s \right\} = \left\{id, \rho, \rho^2, \rho^3 \right\} s$. \checkmark 
				
			\item $\left\{id, \rho^2, s, \rho^2 s\right\}$ is normal as $\rho \left\{id, \rho^2, s, \rho^2 s\right\} = \left\{\rho, \rho^3, \rho s, \rho^3 s \right\} = \left\{id, \rho^2, s, \rho^2 s\right\}\rho$. \checkmark
				
			\item $\left\{id, \rho^2, \rho s, \rho^3 s \right\}$ is normal as $\rho \left\{id, \rho^2, \rho s, \rho^3 s \right\} = \left\{\rho, \rho^3, \rho^2 s, s \right\}$. \checkmark 
				
			\item $D_4$ is normal.
				
			\setcounter{enumi}{1}
			\item The factor group $D_4/\left\{id, \rho^2\right\} = \left\{ \left\{ id, \rho^2\right\}, \left\{\rho, \rho^3\right\}, \left\{s, \rho^2 s\right\}, \left\{\rho s, \rho^3 s\right\} \right\}$. 
			
			\setcounter{enumi}{6}
			\item The factor group $D_4/\left\{id, \rho, \rho^2, \rho^3 \right\} = \left\{ \left\{id, \rho, \rho^2, \rho^3 \right\}, \left\{ s, \rho^3 s, \rho^2 s, \rho s \right\} \right\}$.
			
			\item $D_4/\left\{id, \rho^2, s, \rho^2 s\right\} = \left\{ \left\{id, \rho^2, s, \rho^2 s\right\}, \left\{\rho, \rho^3, \rho s, \rho^3 s \right\}\right\}$.
			
			\item $D_4/\left\{id, \rho^2, \rho s, \rho^3 s \right\} = \left\{\left\{id, \rho^2, \rho s, \rho^3 s \right\},\left\{\rho, \rho^3, \rho^2 s, s \right\}\right\}$. 
			%\end{multicols}
		\end{enumerate}
	\end{proof}

	\setcounter{theorem}{6}
	\begin{theorem}
		Prove or disprove: If $H$ is a normal sub-group of $G$ such that $H$ and $G/H$ are abelian, then $G$ is abelian.
	\end{theorem}
	\begin{proof}[Counterexample]
		Let $G = S_3$ and $H = A_3$. $S_3$ is non-abelian. By Corollary 9.4, $A_3 \simeq \Z_3$, so $A_3$ \textit{is} abelian. We must show $A_3$ is normal and $S_3/A_3$ is abelian:\\
		$A_3$ is normal as for any $\sigma\in S_3$, $\sigma A_3 \inv{\sigma}$ is even whether $\sigma$ is even or odd. So $\sigma A_3 \inv{\sigma} \subseteq A_3$, so $A_3$ is normal by Theorem 10.1.2. \nolinebreak\checkmark\\
		To show $S_3/A_3$ is abelian, notice by Lagrange's Theorem, $\left[S_3 : A_3 \right] = \frac{|S_3|}{|A_3|} = \frac{6}{3} = 2$. By Theorem 10.2, $|S_3/A_3|=\left[S_3 : A_3 \right]$. By Corollary 9.4, since $|S_3/A_3|=2$ and $2$ is prime, $S_3/A_3 \simeq \Z_2$, so $S_3/A_3$ is abelian. \checkmark 
	\end{proof}

	\setcounter{theorem}{10}
	\begin{theorem}
		If a group $G$ has exactly one sub-group $H$ of order $k$, prove that $H$ is normal in $G$.
	\end{theorem}
	\begin{proof}
		By \textbf{Exercise 3.54} from homework \#4, $gH\inv{g}$ is a sub-group of $G$. By the assumption that $H$ is the only sub-group of $G$, we have that $H = gH\inv{g}$. By Theorem 10.1.3, $H = gH\inv{g}\implies H$ is a normal subgroup of $G$. 
	\end{proof}


	\begin{theorem}
		Define the \textit{\textbf{centralizer}} of an element $g$ in a group $G$ to be the set \[C\left(g\right) = \left\{x\in G : xg = gx \right\}\text{.}\] Show that $C(g)$ is a sub-group of $G$. If $g$ generates a normal sub-group of $G$, prove that $C(g)$ is normal in $G$. 
	\end{theorem}
	\begin{proof}
		For $C(g)\subseteq G$ to be a sub-group of $G$, it is sufficient to show 
		\begin{enumerate}
		\begin{multicols}{2}
			\item For all $a,b\in C(g)$, $a\circ b\in C(g)$.
			
			\item There exists $e\in C(g)$ such that $a\circ e=a=e\circ a$ for all $a\in C(g)$. 
			
			\item For all $a\in C(g)$ there exists $\inv{a}\in C(g)$ such that $a\circ\inv{a}=e=\inv{a}\circ a$.
			
			\setcounter{enumi}{0}
		\end{multicols}
			\item Consider $a,b\in C(g)$. Then $a,b\in G$ as $C(g)\subseteq G$. Then \begin{dmath*}
				(ab)x = a(bx) \condition[]{by associativity of elements of $G$} = a(xb) \condition[]{by assumption that $b\in C(g)$} = (ax)b \condition[]{by associativity of elements of $G$} = (xa)b \condition[]{by assumption that $a\in C(g)$} = x(ab) \condition[]{by associativity of elements of $G$}
			\end{dmath*} So $ab\in C(g)$. \checkmark 
		
			\item Because $e\in G$ by definition commutes with every element of $G$, $e\in C(g)$. \checkmark 
			
			\item Consider $c\in C(g)$. Then $c\in G$ and $\inv{c}\in G$ as $G$ is a group and $C(g) \subseteq G$. Then \begin{dmath*}
				c \hiderel{\in} C(g) \implies cx \hiderel{=} xc \implies \inv{c}cx\inv{c} \hiderel{=} \inv{c}xc\inv{c} \condition[]{by left and right multiplying by $\inv{c}$} \implies x\inv{c} \hiderel{=} \inv{c}x \condition[]{by condensing the ``$\inv{c}c$" and ``$c\inv{c}$" terms}
			\end{dmath*} So $c\in C(g)\implies \inv{c} \in C(g)$. \checkmark
		\end{enumerate}
		So $C(g)$ is a sub-group of $G$. \\
		Because $C(g)$ us clearly abelian, it follows that the left and right co-sets must be equal, and $C(g)$ must be normal.
	\end{proof}


	\setcounter{section}{11}
	\setcounter{theorem}{1}
	\begin{theorem}
		Which of the following maps are homomorphisms? If the map is a homomorphism, what is the kernel?
		\begin{enumerate}[(a)]
			\item $\phi:\R^*\to GL_2\left(\R\right)$ defined by \[\phi(a)=\begin{bmatrix}1&0\\0&a\end{bmatrix}\]
			
			\item $\phi:\R\to GL_2\left(\R\right)$ defined by \[\phi(a)=\begin{bmatrix}1&0\\a&1\end{bmatrix}\]
			
			\item $\phi:GL_2\left(\R\right)\to\R$ defined by \[\phi\left(\begin{bmatrix} a&b\\c&d \end{bmatrix}\right)=a+d\]
			
			\item $\phi:GL_2\left(\R\right)\to\R^*$ defined by \[\phi\left(\begin{bmatrix} a&b\\c&d \end{bmatrix}\right)=ad-bc\]
			
			\item $\phi:\mathbb{M}_2\left(\R\right)\to\R$ defined by \[\phi\left(\begin{bmatrix} a&b\\c&d \end{bmatrix}\right)=b\]
		\end{enumerate}
	\end{theorem}
	\begin{proof}
		\hfill 
		\begin{enumerate}[(a)]
			\item $\phi$ is a homomorphism as for $a,b\in \R^*$, \[
			\phi(a)\phi(b) = \begin{bmatrix} 1 & 0 \\ 0 & a \end{bmatrix} \begin{bmatrix} 1 & 0 \\ 0 & b \end{bmatrix} = \begin{bmatrix} 1 & 0 \\ 0 & ab \end{bmatrix} = \phi(ab)
			\] and $\ker\phi := \left\{x\in\R^* \text{ such that } \phi(x)=id \right\} = \left\{1\right\}$. \qed
			
			\item $\phi$ is not a homomorphism as \[
			\phi(a)\phi(b) = \begin{bmatrix} 1 & 0 \\ a & 1 \end{bmatrix} \begin{bmatrix} 1 & 0 \\ b & 1 \end{bmatrix} = \begin{bmatrix} 1 & 0 \\ a+b & 1 \end{bmatrix} \not= \phi(ab)\text{.}\]\hfill\qed
			
			\item $\phi$ is not a homomorphism as \[
			\phi\left(\begin{bmatrix} a & b \\ c & d \end{bmatrix} \begin{bmatrix} \alpha & \beta \\ \gamma & \delta \end{bmatrix}\right) = \phi\left(\begin{bmatrix} a\alpha + b\gamma & a\beta + b\delta \\ c\alpha + d\gamma & c\beta + d\delta \end{bmatrix}\right) = a\alpha + b\gamma + c\beta + d\delta \not= a\alpha + a\delta + d\alpha + d\delta = \phi\left(\begin{bmatrix} a & b \\ c & d \end{bmatrix}\right) \phi\left(\begin{bmatrix} \alpha & \beta \\ \gamma & \delta \end{bmatrix}\right)\text{.}
			\]
			
			\item $\phi$ is a homomorphism as \begin{dgroup*}\begin{dmath*}
			\phi\left(\begin{bmatrix} a & b \\ c & d \end{bmatrix} \begin{bmatrix} \alpha & \beta \\ \gamma & \delta \end{bmatrix}\right) \hiderel{=} \phi\left(\begin{bmatrix} a\alpha + b\gamma & a\beta + b\delta \\ c\alpha + d\gamma & c\beta + d\delta \end{bmatrix}\right) = (a\alpha + b\gamma)(c\beta + d\delta) - (a\beta+b\delta)(c\alpha+d\gamma) \\ = ac\alpha\beta + ad\alpha\delta + bc\beta\gamma + bd\gamma\delta - ac\alpha\beta - ad\beta\gamma - bc\alpha\delta - bd\gamma\delta = ad\alpha\delta + bc\beta\gamma - ad\beta\gamma - bc\gamma\delta
			\end{dmath*}\begin{dmath*}
			= (ad - bc) (\alpha\delta - \beta\gamma) \hiderel{=} \phi\left(\begin{bmatrix} a & b \\ c & d \end{bmatrix}\right) \phi\left(\begin{bmatrix} \alpha & \beta \\ \gamma & \delta \end{bmatrix}\right)
			\end{dmath*}
			\end{dgroup*} and $\ker\phi := \left\{M\in GL_2\left(\R\right) \text{ such that } \phi(A)=1\right\} = SL_2\left(\R\right)$. \qed
		
			\item $\phi$ is not a homomorphism as \[
			\phi\left(\begin{bmatrix} a & b \\ c & d \end{bmatrix} \begin{bmatrix} \alpha & \beta \\ \gamma & \delta \end{bmatrix}\right) \hiderel{=} \phi\left(\begin{bmatrix} a\alpha + b\gamma & a\beta + b\delta \\ c\alpha + d\gamma & c\beta + d\delta \end{bmatrix}\right) = a\beta + b\delta \not= b\beta = \phi\left(\begin{bmatrix} a & b \\ c & d \end{bmatrix}\right) \phi\left(\begin{bmatrix} \alpha & \beta \\ \gamma & \delta \end{bmatrix}\right)\text{.}\qedhere
			\]
		\end{enumerate}
	\end{proof}

	\setcounter{theorem}{16}
	\begin{theorem}
		If $H$ and $K$ are normal sub-groups of $G$ and $H\cap K = \left\{e\right\}$, prove that $G$ is isomorphic to a sub-group of $G\setminus H\times G\setminus K$.
	\end{theorem}
	\begin{proof}
		content...
	\end{proof}

	\begin{singlespace}
	\rule{\textwidth}{.5pt}
	Homework exercises I cited:
	\begin{description}
		\item [Exercise 3.7] Let $S=\mathbb{R}\setminus\left\{-1\right\}$ and define a binary operation on $S$ by $a\ast b=a+b+ab$. Prove that $\left(S,\ast\right)$ is an abelian group.
		An \textit{abelian group} is a group $G$ such that $a\ast b=b\ast a$ for all $a,b\in G$. 
		
		\begin{description}
			\item[Associative] For all $a,b,c\in G$, $(a\ast b)\ast c=a\ast(b\ast c)$.\\
			\begin{dmath*}
				(a\ast b)\ast c=(a\ast b)+c+(a\ast b)c\condition[]{by definition of $a\ast b$} = (a+b+ab)+c+(a+b+ab)c\condition[]{by definition of $a\ast b$} = a+b+c+ab+ac+bc+abc=a+(b+c+bc)+a(b+c+bc)=a+(b\ast c)+a(b\ast c)\condition[]{by definition of $a\ast b$} = a\ast(b\ast c)\condition[]{by definition of $a\ast b$}
			\end{dmath*}
			\item[Identity element] There exists an element $e\in G$ such that for any $a\in G$, $e\ast a=a\ast e=a$.\\
			For any $a$, let $b=0$. Then $a\ast b=a+0+a(0)=a=0+a+0(a)=b\ast a$. So $b=0$ is the identity element such that $a\ast0=0\ast a$ for all $a\in G$. 
			\item[Inverse element] For each element $a\in G$ there exists an $a^{-1}\in G$ such that $a\ast a^{-1}=a^{-1}\ast a=e$.\\ 
			We know from above that $e=0$. So given $a\in G$,
			\begin{dmath*}
				a+b+ab=0 \implies b(1+a)+a\hiderel{=}0 \implies b\hiderel{=}\frac{-a}{1+a}
			\end{dmath*} which is defined for all $x\in S$. So $b=\frac{-a}{1+a}$ is the unique inverse element $a^{-1}$ to each $a$ such that $a\ast a^{-1}=a^{-1}\ast a=e$.
			\item[Commutative] For all $a,b\in G$, $a\ast b=b\ast a$. 
			\begin{dmath*}
				a\ast b=a+b+ab=b+a+ab\condition[]{by commutative property of addition} = b+a+ba\condition[]{by commutative property of multiplication}=b\ast a\condition[]{by definition}
			\end{dmath*}
			
		\end{description}
		So $(S,\ast)$ is an abelian group. \qed 
	\end{description}
	
	\setcounter{section}{3}
	\setcounter{theorem}{53}
	\begin{theorem}
		Let $H$ be a sub-group of $G$. If $g\in G$, show that $gH\inv{g}:=\left\{\inv{g}hg: h\in H\right\}$ is also a sub-group of $G$. 
	\end{theorem}
	\begin{proof}
		By theorem from class, for $gH\inv{g}\subseteq G$ to be a sub-group of $G$, it is sufficient to show 
		\begin{enumerate}
			\item For all $a,b\in gH\inv{g}$, $a\circ b\in gH\inv{g}$.
			
			\item There exists $e\in gH\inv{g}$ such that $a\circ e=a=e\circ a$ for all $a\in gH\inv{g}$. 
			
			\item For all $a\in gH\inv{g}$ there exists $\inv{a}\in gH\inv{g}$ such that $a\circ\inv{a}=e=\inv{a}\circ a$.
		\end{enumerate}
		Notice that $gH\inv{g}$ is necessarily a subset of $G$ as every element in $H$ is contained in $G$ (by assumption that $H$ is a sub-group of $G$). So $g,h,\inv{g}\in G$. Furthermore, every element in $gH\inv{g}$ is of the form $\inv{g}hg$, and $G$ is closed by assumption that $G$ is a group. So $gH\inv{g}\subseteq G$.\\
		Let $a,b\in gH\inv{g}$. Then $a=\inv{g}h_ag$ and $b=\inv{g}h_bg$ for some $h_a,h_b\in H$. 
		\begin{enumerate}
			\item Consider \begin{dmath*}
				ab = \left(\inv{g}h_ag\right)\left(\inv{g}h_bg\right) = \left(\inv{g}h_a\right)\left(g\inv{g}\right)\left(h_bg\right) \condition[]{by associativity of elements of $G$} = \left(\inv{g}h_a\right)\left(e\right)\left(h_bg\right) \condition[]{by definition of $\inv{g}$} = \left(\inv{g}h_a\right)\left(h_bg\right) \condition{by definition of $e$} = \inv{g}\left(h_ah_b\right)g \condition[]{by associativity of elements of $G$}
			\end{dmath*} and $\left(h_ah_b\right)\in H$ as $H$ was assumed to be a sub-group, so $H$ is closed. So $ab=\inv{g}\left(h_ah_b\right)g$ is of the form $\inv{g}hg$ for some $h\in H$. So $gH\inv{g}$ is closed. 
			
			\item By assumption that $H$ is a sub-group of $G$, $e\in H$. So $\left(\inv{g}eg\right)\in gH\inv{g}$ and \begin{dmath*}
				\inv{g}eg = \inv{g}g \condition[]{by definition of $e$} = e \condition{by definition of $\inv{g}$}.
			\end{dmath*} So $\left(\inv{g}eg\right)\in gH\inv{g}$ and $\inv{g}eg=e$. so $e\in gH\inv{g}$. 
			
			\item By Proposition 3.4, if $a=\inv{g}h_ag$ then $\inv{a}=\inv{g}\inv{h_a}g$. So $\inv{a}\in gH\inv{g}$ if $\inv{h_a}\in H$, and $\inv{h_a}$ is necessarily an element of $H$ by assumption that $H$ is a sub-group of $G$. So $a\in gH\inv{g}\implies \inv{a}\in gH\inv{g}$. 
		\end{enumerate} So this shows that $gH\inv{g}$ is a sub-group of $G$. 
	\end{proof}

	\end{singlespace}
\end{document}